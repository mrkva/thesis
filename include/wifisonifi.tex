\documentclass{article}

\usepackage[latin1]{inputenc}
\usepackage{tikz}
\usetikzlibrary{shapes,arrows,fit,backgrounds}
\begin{document}
\pagestyle{empty}

\tikzstyle{block} = [rectangle, draw, 
    text width=7em, text centered, minimum height=2.5em, node distance=3cm,
	top color=white, % a shading that is white at the top...
	    bottom color=black!10] % and something else at the bottom]
\tikzstyle{line} = [draw, -triangle 90]


\begin{tikzpicture}[node distance = 2cm, auto]
    % Place nodes
    \node [block] (nds) {\texttt{Network data stream}};
	\node [block, above of=nds] (rni) {\texttt{Router, other network users}};
    \node [block, right of=nds] (sniffer) {\texttt{Packet sniffer}};
	\node [block, right of=sniffer] (aplay) {\texttt{Aplay - actual sonification}};
	\node [block, below of=sniffer] (me) {\texttt{Me}};
	\node [block, right of=aplay] (aout) {\texttt{Audio output}};

    % Draw edges
    \path [line] (rni) -- (nds);
	\path [line] (nds) -- (sniffer);
	\path [line] (sniffer) -- (aplay);
	\path [line] (aplay) -- (aout);
	\path [line, rounded corners=5pt] (me) -| node [near end, above, scale=.9, rotate=90] {\texttt{Arbitrary data}} (nds);
	\path [line, rounded corners=5pt] (me) -| node [near end, below, scale=.8, rotate=90] {\texttt{\begin{tabular}{l}
	Sonification\\ parameters 
	\end{tabular}}} (aplay);
	
	\begin{scope}[on background layer]
	    \node[fill=lightgray!50,inner sep = 4mm,fit=(aplay)(sniffer),label=above:\texttt{Computer}] {}; 
	\end{scope}
	
\end{tikzpicture}

\end{document}